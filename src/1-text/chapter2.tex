\chapter{Основные команды}

В этом разделе познакомимся с основными командами, которые могут пригодиться.

\section{Цитирование и ссылка}
Цитированные работы нужно вставить в формате BibTeX в файл е \textit{src/2-bib/refs.bib}

Большинство академических сайтов (google scholar, iee и др.) поддерживают такой формат цитирования. Для сайтов, которые не поддерживают BibTeX, в папке \textit{src/2-bib/template} лежат шаблоны для самостоятельного внесения источника.

Запись BibTeX выглядит как показано ниже\\[0.25cm]


@type\{id,

...

\}\\[0.25cm]


Для цитирования используется команда \textbf{cite\{id\}}

Например, Lorem ipsum dolor sit amet \text{cite\{pobedryaMSS\}}

Lorem ipsum dolor sit amet \cite{pobedryaMSS}

Также можно ссылаться на уравнения и теоремы, \textbf{eqref\{id\}, ref\{id\}} соответсвенно.

Запись теоремы выглядит как показано ниже\\[0.25cm]


createtheorem\;\{id\}\;\{text\}\\[0.25cm]



Например, В теореме \textbf{ref\{th:vitae\}} используетя \textbf{eqref\{eq:st:3\}} и 
\textbf{eqref\{ch-2:eq:st:7\}}

В теореме \ref{th:vitae} используетя \eqref{eq:st:3} и \eqref{ch-2:eq:st:7}

\createeq{eq:st:3}{E = mc^2}

\clearpage
\section{Макросы}
Находятся в папке \textit{settngs/macros}

\subsection{Кратинки}
Команда \textbf{image\{PATH\_TO\_IMG\}\{Название\}\{label\}\{scale\}}
\image{src/img/logo.jpg}{Символика МАИ}{fg:st:plate}{0.35}

\subsection{Таблицы}
Команда \textbf{createtable\{название таблицы\}\{label\}\{каркас таблицы\}} 
\createtable
{Анализ роста числа параметров при развитии моделей глубокого обучения}
{tb:intro:1}
{
    \begin{tabular}{|l|l|l|l|l|l|l|}
    \hline
    Название               & AlexNet     & VGGNet      & ResNet      & BERT     & mT5   & GPT3  \\ \hline
    Год                          & 2012        & 2014        & 2015        & 2018     & 2020  & 2020  \\ \hline
    Тип данных             & изображение & изображение & изображение & текст    & текст & текст \\ \hline
    Число параметров, млрд & $0{,}06$    & $0{,}13$    & $0{,}06$    & $0{,}34$ & $13$  & $175$ \\ \hline
    \end{tabular}
}

\subsection{Уравнения и теоремы}
Команда \textbf{createeq or createtheorem\{label\}\{text\}}

\subsection{Глава без нумерации}
Команда \textbf{chapterbul\{Название главы\}\{ссылка на *.tex\}}